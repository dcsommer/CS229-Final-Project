% THIS IS SIGPROC-SP.TEX - VERSION 3.1
% WORKS WITH V3.2SP OF ACM_PROC_ARTICLE-SP.CLS
% APRIL 2009
%
% It is an example file showing how to use the 'acm_proc_article-sp.cls' V3.2SP
% LaTeX2e document class file for Conference Proceedings submissions.
% ----------------------------------------------------------------------------------------------------------------
% This .tex file (and associated .cls V3.2SP) *DOES NOT* produce:
%       1) The Permission Statement
%       2) The Conference (location) Info information
%       3) The Copyright Line with ACM data
%       4) Page numbering
% ---------------------------------------------------------------------------------------------------------------
% It is an example which *does* use the .bib file (from which the .bbl file
% is produced).
% REMEMBER HOWEVER: After having produced the .bbl file,
% and prior to final submission,
% you need to 'insert'  your .bbl file into your source .tex file so as to provide
% ONE 'self-contained' source file.
%
% Questions regarding SIGS should be sent to
% Adrienne Griscti ---> griscti@acm.org
%
% Questions/suggestions regarding the guidelines, .tex and .cls files, etc. to
% Gerald Murray ---> murray@hq.acm.org
%
% For tracking purposes - this is V3.1SP - APRIL 2009

\documentclass{acm_proc_article-sp}
\usepackage{amsmath, amsthm, amssymb, fancyhdr, enumerate}

\begin{document}

\title{Spike Sorting using Machine Learning}
\subtitle{Using unsupervised algorithms to automatically classify cells}
%
% You need the command \numberofauthors to handle the 'placement
% and alignment' of the authors beneath the title.
%
% For aesthetic reasons, we recommend 'three authors at a time'
% i.e. three 'name/affiliation blocks' be placed beneath the title.
%
% NOTE: You are NOT restricted in how many 'rows' of
% "name/affiliations" may appear. We just ask that you restrict
% the number of 'columns' to three.
%
% Because of the available 'opening page real-estate'
% we ask you to refrain from putting more than six authors
% (two rows with three columns) beneath the article title.
% More than six makes the first-page appear very cluttered indeed.
%
% Use the \alignauthor commands to handle the names
% and affiliations for an 'aesthetic maximum' of six authors.
% Add names, affiliations, addresses for
% the seventh etc. author(s) as the argument for the
% \additionalauthors command.
% These 'additional authors' will be output/set for you
% without further effort on your part as the last section in
% the body of your article BEFORE References or any Appendices.

\numberofauthors{3} %  in this sample file, there are a *total*
% of EIGHT authors. SIX appear on the 'first-page' (for formatting
% reasons) and the remaining two appear in the \additionalauthors section.
%
\author{
% You can go ahead and credit any number of authors here,
% e.g. one 'row of three' or two rows (consisting of one row of three
% and a second row of one, two or three).
%
% The command \alignauthor (no curly braces needed) should
% precede each author name, affiliation/snail-mail address and
% e-mail address. Additionally, tag each line of
% affiliation/address with \affaddr, and tag the
% e-mail address with \email.
%
% 1st. author
\alignauthor
David Brody\\
       \email{dbrody@stanford.edu}
% 2nd. author
\alignauthor Daniel Sommermann\\
       \email{dcsommer@stanford.edu}
% 3rd. author
\alignauthor
Michael Gummelt\\
       \email{mgummelt@stanford.edu}
}
\date{16 December 2011}
% Just remember to make sure that the TOTAL number of authors
% is the number that will appear on the first page PLUS the
% number that will appear in the \additionalauthors section.

\maketitle
\begin{abstract}
In this paper, we outline our approach to automatic spike sorting in the
context of electrical response from neurons. Using compression techniques
such as PCA and polynomial approximations, we were able to find a
tractable subspace in which to analyze our features. On our reduced
dimension samples, we then applied two algorithms wrapping k-means which
help find k: g-means clustering and using the Davies-Bouldin index. We
were able to get some 
very good clusters while other clusters were not as well separated,
indicating that there was a combination of a large degree of noise in our
features and high variability in terms of the response of each sensor when
a single cell spikes.

\end{abstract}

\section{Introduction}
The equipment gathers voltage samples at a rate of 10 kHz. For our
experiments, we recorded 100 seconds of data. Imprecisely, we have heard
from the lab that each cell fires about 10 times a second, and that the
equipment can reliably discern 30 cells so we expect
each cluster to have around 3000 elements on average.

\section{The {\secit Body} of The Paper}
Typically, the body of a paper is organized
into a hierarchical structure, with numbered or unnumbered
headings for sections, subsections, sub-subsections, and even
smaller sections.  The command \texttt{{\char'134}section} that
precedes this paragraph is part of such a
hierarchy.\footnote{This is the second footnote.  It
starts a series of three footnotes that add nothing
informational, but just give an idea of how footnotes work
and look. It is a wordy one, just so you see
how a longish one plays out.} \LaTeX\ handles the numbering
and placement of these headings for you, when you use
the appropriate heading commands around the titles
of the headings.  If you want a sub-subsection or
smaller part to be unnumbered in your output, simply append an
asterisk to the command name.  Examples of both
numbered and unnumbered headings will appear throughout the
balance of this sample document.

Because the entire article is contained in
the \textbf{document} environment, you can indicate the
start of a new paragraph with a blank line in your
input file; that is why this sentence forms a separate paragraph.

\subsection{Dimensionality reduction and compression}
\subsubsection{PCA}
At this point, we have our features and are ready to run unsupervised
learning algorithms on the data set. However, as we began experimenting
with our data set, we realized that the time needed to learn on our data
set was simply too great. We needed a way to reduce the dimensionality of
our features. Since a feature vector consists of 30 measurements for 60
channels each, we have features in $\mathbb{R}^{1800}$.

The first approach we took was to apply PCA and take the top 300 principal
components and project our features into that space, $\mathbb{R}^{300}$.

Performing this optimization allowed us to run k-means clustering in 13\%
of the time without compression. We experimented with projecting into
different dimension subspaces and found 300 to be a good dimension that
saves enough on running time without sacrificing too much in terms of
underfitting.

\begin{figure}
\centering
\includegraphics[width=3in,height=2in]{../poster/images/dim_vs_runningtime.png} 
\caption{Here we plot the time clustering took (using g-means) vs the
  dimension we reduced to using PCA. The roughly linear relationship shows
how the tradeoff between dimensionality of our feature vector and running
time was straightforward.}
\end{figure}

Here we can clearly see the relationship between the dimensionality of our
featuer vectors and the time it takes to run our clustering code. As we
iterated on our project, reducing the running time of our algorithm was
critical to being able to try many different ideas quickly.

\subsubsection{Fitting Polynomials}
We also tried fitting polynomials of various degrees. In this case, we let
our feature vector be in $\mathbb{R}^{60(n+1)}$, representing the coefficients to
60 polynomials (one for each channel), each of degree $n$.

<figures of dimension vs running time, performance>

\subsection{Clustering Algorithms}
--Gummelt intro

\subsubsection{G-Means algorithm}
One way to choose k in the k-means algorithm is to use an algorithm
developed in 2003 by Greg Hamerly and Charles Elka, presented in the NIPS
2003 conference \cite{gmeans}. In this method, a small value for k is
chosen and kmeans is run. Then, for each cluster produced, we use a
heuristic function to decide whether to split that cluster. The heuristic
used is the Anderson-Darling statistic. If that statistic is greater than
some threshold critical value, then we decide that the cluster should be
split. The Anderson-Darling statistic is computed as follows:
$$
A^2(Z) = -n - \frac{1}{n}\sum_{i=1}^n (2i -
1)[\log(z_i)+\log(1-z_{n+1-i})]
$$$$
A^2_*(Z) = A^2(Z)(1 + 4/n - 25/n^2)
$$
If the $A^2_*(Z)$ is greater than the threshold, we split. For this
project we experimented with many thresholds and found that with $t=12$ we
got good separation between clusters.

\begin{figure}
\centering
\includegraphics[width=3in,height=2in]{../poster/images/gmeans_k_vs_metric.png} 
\caption{Here we plot the average Anderson-Darling statistic across all
clusters as a function of k. We can see how we arrive at a value of k
to use when the statistic crosses the threshold (12)}
\end{figure}

\subsection{Theorem-like Constructs}
Other common constructs that may occur in your article are
the forms for logical constructs like theorems, axioms,
corollaries and proofs.  There are
two forms, one produced by the
command \texttt{{\char'134}newtheorem} and the
other by the command \texttt{{\char'134}newdef}; perhaps
the clearest and easiest way to distinguish them is
to compare the two in the output of this sample document:

This uses the \textbf{theorem} environment, created by
the\linebreak\texttt{{\char'134}newtheorem} command:
\newtheorem{theorem}{Theorem}
\begin{theorem}
Let $f$ be continuous on $[a,b]$.  If $G$ is
an antiderivative for $f$ on $[a,b]$, then
\begin{displaymath}\int^b_af(t)dt = G(b) - G(a).\end{displaymath}
\end{theorem}

The other uses the \textbf{definition} environment, created
by the \texttt{{\char'134}newdef} command:
\newdef{definition}{Definition}
\begin{definition}
If $z$ is irrational, then by $e^z$ we mean the
unique number which has
logarithm $z$: \begin{displaymath}{\log e^z = z}\end{displaymath}
\end{definition}

\begin{figure}
\centering
\includegraphics[width=1in,height=1in]{sns.jpg}
\caption{A sample black and white graphic (.ps format) that has
been resized with the \texttt{psfig} command.}
\end{figure}

Two lists of constructs that use one of these
forms is given in the
\textit{Author's  Guidelines}.

\begin{figure*}
\centering
\includegraphics[width=1in,height=1in]{sns.jpg}
\caption{A sample black and white graphic (.eps format)
that needs to span two columns of text.}
\end{figure*}
and don't forget to end the environment with
{figure*}, not {figure}!
 
There is one other similar construct environment, which is
already set up
for you; i.e. you must \textit{not} use
a \texttt{{\char'134}newdef} command to
create it: the \textbf{proof} environment.  Here
is a example of its use:
\begin{proof}
Suppose on the contrary there exists a real number $L$ such that
\begin{displaymath}
\lim_{x\rightarrow\infty} \frac{f(x)}{g(x)} = L.
\end{displaymath}
Then
\begin{displaymath}
l=\lim_{x\rightarrow c} f(x)
= \lim_{x\rightarrow c}
\left[ g{x} \cdot \frac{f(x)}{g(x)} \right ]
= \lim_{x\rightarrow c} g(x) \cdot \lim_{x\rightarrow c}
\frac{f(x)}{g(x)} = 0\cdot L = 0,
\end{displaymath}
which contradicts our assumption that $l\neq 0$.
\end{proof}

Complete rules about using these environments and using the
two different creation commands are in the
\textit{Author's Guide}; please consult it for more
detailed instructions.  If you need to use another construct,
not listed therein, which you want to have the same
formatting as the Theorem
or the Definition\cite{salas:calculus} shown above,
use the \texttt{{\char'134}newtheorem} or the
\texttt{{\char'134}newdef} command,
respectively, to create it.

\subsection*{A {\secit Caveat} for the \TeX\ Expert}
Because you have just been given permission to
use the \texttt{{\char'134}newdef} command to create a
new form, you might think you can
use \TeX's \texttt{{\char'134}def} to create a
new command: \textit{Please refrain from doing this!}
Remember that your \LaTeX\ source code is primarily intended
to create camera-ready copy, but may be converted
to other forms -- e.g. HTML. If you inadvertently omit
some or all of the \texttt{{\char'134}def}s recompilation will
be, to say the least, problematic.

\section{Conclusions}
This paragraph will end the body of this sample document.
Remember that you might still have Acknowledgments or
Appendices; brief samples of these
follow.  There is still the Bibliography to deal with; and
we will make a disclaimer about that here: with the exception
of the reference to the \LaTeX\ book, the citations in
this paper are to articles which have nothing to
do with the present subject and are used as
examples only.
%\end{document}  % This is where a 'short' article might terminate

%ACKNOWLEDGMENTS are optional
\section{Acknowledgments}
This section is optional; it is a location for you
to acknowledge grants, funding, editing assistance and
what have you.  In the present case, for example, the
authors would like to thank Gerald Murray of ACM for
his help in codifying this \textit{Author's Guide}
and the \textbf{.cls} and \textbf{.tex} files that it describes.

%
% The following two commands are all you need in the
% initial runs of your .tex file to
% produce the bibliography for the citations in your paper.
\bibliographystyle{abbrv}
\bibliography{sigproc}  % sigproc.bib is the name of the Bibliography in this case
% You must have a proper ".bib" file
%  and remember to run:
% latex bibtex latex latex
% to resolve all references
%
% ACM needs 'a single self-contained file'!
%
%APPENDICES are optional
%\balancecolumns
\appendix
%Appendix A
\section{Headings in Appendices}
The rules about hierarchical headings discussed above for
the body of the article are different in the appendices.
In the \textbf{appendix} environment, the command
\textbf{section} is used to
indicate the start of each Appendix, with alphabetic order
designation (i.e. the first is A, the second B, etc.) and
a title (if you include one).  So, if you need
hierarchical structure
\textit{within} an Appendix, start with \textbf{subsection} as the
highest level. Here is an outline of the body of this
document in Appendix-appropriate form:
\subsection{Introduction}
\subsection{The Body of the Paper}
\subsubsection{Type Changes and  Special Characters}
\subsubsection{Math Equations}
\paragraph{Inline (In-text) Equations}
\paragraph{Display Equations}
\subsubsection{Citations}
\subsubsection{Tables}
\subsubsection{Figures}
\subsubsection{Theorem-like Constructs}
\subsubsection*{A Caveat for the \TeX\ Expert}
\subsection{Conclusions}
\subsection{Acknowledgments}
\subsection{Additional Authors}
This section is inserted by \LaTeX; you do not insert it.
You just add the names and information in the
\texttt{{\char'134}additionalauthors} command at the start
of the document.
\subsection{References}
Generated by bibtex from your ~.bib file.  Run latex,
then bibtex, then latex twice (to resolve references)
to create the ~.bbl file.  Insert that ~.bbl file into
the .tex source file and comment out
the command \texttt{{\char'134}thebibliography}.
% This next section command marks the start of
% Appendix B, and does not continue the present hierarchy
\section{More Help for the Hardy}
The acm\_proc\_article-sp document class file itself is chock-full of succinct
and helpful comments.  If you consider yourself a moderately
experienced to expert user of \LaTeX, you may find reading
it useful but please remember not to change it.
\balancecolumns
% That's all folks!
\end{document}
