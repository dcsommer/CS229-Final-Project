We use k means to determine the optimal clusters for a particular
value of k, but we must use a separate technique for determining the
optimal value of k, or the number of cells in the system.  We use two
such techniques and compare their efficacy.  The first is the
Davies-Bouldin index.  It favors sets of clusters with high
intra-cluster similarity and low inter-cluster similarity.  The
formula is as follows:

$X_j$ is sample $j$
$A_i$ is the center of cluster $i$
$T_i$ is the number of samples assigned to cluster $i$

$S_i$ measures the intra cluster similarity of cluster $i$

$S_i = (\frac{1}{T_i} \sum\limits_{j=1}^{T_i} |X_j - A_i|^q)^{1\q)$

$M_{ij}$ measures the inter cluster similarity of clusters $i$ and $j$

$M_{ij} = (\sum\limits_{k=1}^{N} |a_{ki} - a{kj}|^p)^{1/p}$

$DB$ is the Davies Bouldin index

$DB = \frac{1}{N} \sum\limits_{i=1}^N \max_{j:i \neq j} R_{ij}$

The following is a graph of k vs DB:

..graph..















